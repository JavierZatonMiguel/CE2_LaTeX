\documentclass[a4paper,12pt]{article}

\usepackage{CustomPacks}

\usepackage{titlesec}
\titlelabel{\thetitle\enspace}

\begin{document}


%%%%%%%%%%%%%%%%%%%%%%%%%%%%%%%%%%% PORTADA %%%%%%%%%%%%%%%%%%%%%%%%%%%%%%%%%%%%%%%%%
\begin{center}
\thispagestyle{empty}

\includegraphics[width=\textwidth]{Logo_portada.png} \\
\vspace{0.5cm}
{\large
Universidad Politécnica de Madrid \\
Instituto Universitario de Microgravedad “Ignacio Da Riva” \\
Máster Universitario en Sistemas Espaciales}\\ 
\vspace{0.5cm}
{\large Caso de Estudio 2}\\ 
\vspace{1.5cm} % 1.75

{\huge \textbf{Modelización térmica del Telescope Assembly de la misión ARIEL mediante ESATAN-TMS}}\\ 
\vspace{0.5cm}

\includegraphics[width= 7cm]{IM/ARIEL-LOGO.png}

\vspace{1cm} % 6.5
\begin{center}
        \vspace{0.5cm}
        {\large Zatón Miguel, Javier } \\ \href{mailto:javier.zmiguel@alumnos.upm.es} {javier.zmiguel@alumnos.upm.es} \vspace{1.5cm}

       % {\large Profesora: Elena Roibás} \\
       % \href{mailto:ignacio.torralbo@upm.es} {elena.roibas@upm.es} \hspace{0.5cm}
\vspace{0.5cm}
\end{center}
\end{center}


\vspace{0.5cm}

\begin{center}
    \large{\textbf{\today}}
\end{center}
%%%%%%%%%%%%%%%%%%%%%%%%%%%%%%%%%%%%%%%%%%%%%%%%%%%%%%%%%%%%%%%%%%%%%%%%%%%%%%%%%%%%%
%%%%%%%%%%%%%%%%%%%%%%%%%%%%%%%%%%%%%CABECERO%%%%%%%%%%%%%%%%%%%%%%%%%%%%%%%%%%%%%%%%
%%%%%%%%%%%%%%%%%%%%%%%%%%%%%%%%%%%%%%%%%%%%%%%%%%%%%%%%%%%%%%%%%%%%%%%%%%%%%%%%%%%%%
\newpage
\rhead[]{CE2}
\thispagestyle{empty}
%\tableofcontents

\newpage
\rhead[]{CE2}
%\thispagestyle{empty}
%\listoffigures
%\listoftables
\newpage

%%%%%%%%%%%%%%%%%%%%%%%%%%%%%%%%%%%%%%%%%%%%%%%%%%%%%%%%%%%%%%%%%%%%%%%%%%%%%%%%%%%%%
%%%%%%%%%%%%%%%%%%%%%%%%%%%%%%%%%%%%%INDICES y GLOSARIO%%%%%%%%%%%%%%%%%%%%%%%%%%%%%%%%%%%%%%%%
%%%%%%%%%%%%%%%%%%%%%%%%%%%%%%%%%%%%%%%%%%%%%%%%%%%%%%%%%%%%%%%%%%%%%%%%%%%%%%%%%%%%%
\newpage
% \thispagestyle{empty}

\pagenumbering{Roman} % Los índices y acrónimos se numerarán con números romanos

\tableofcontents

\newpage

\listoffigures

\newpage

\listoftables

\newpage
    \printglossary[type=\acronymtype]
\newpage

\newpage
%%%%%%%%%%%%%%%%%%%%%%%%%%%%%%%%%%%%%%%%%%%%%%%%%%%%%%%%%%%%%%%%%
%%%%%%%%%%%%%%%%%%% CONTENIDO DEL INFORME %%%%%%%%%%%%%%%%%%%%%%%
%%%%%%%%%%%%%%%%%%%%%%%%%%%%%%%%%%%%%%%%%%%%%%%%%%%%%%%%%%%%%%%%%
\pagenumbering{arabic}

\setcounter{page}{1} % Para que empiece a contar en 1
\rhead[]{CE2}


%Comandos para usar esto \acrlong{gcd},  \acrshort{adcs}. \acrfull{lcm}.
%Creo que hay varias opciones para que ponga los acrónimos, o bien ssolo cuando aparecen y en qué página lo han hecho o bien por orden alfabético, se usen o no (lo que queráis)

%\makeglossaries{}
\newacronym{tob}{TOB}{Telescope Optical Bench}
\newacronym{svm}{SVM}{Service Warm Module}
\newacronym{plm}{PLM}{Payload Module}
\newacronym{ta}{TA}{Telescope Assembly}
\newacronym{airs}{AIRS}{Ariel Infrated-Red Spectometer}
\newacronym{fgs}{FGS}{Fine Guidance System}

%Comandos para usar esto \acrlong{gcd},  \acrshort{adcs}. \acrfull{lcm}.
%Creo que hay varias opciones para que ponga los acrónimos, o bien ssolo cuando aparecen y en qué página lo han hecho o bien por orden alfabético, se usen o no (lo que queráis)

\section{A} 
\subsection{B}
\acrshort{acr} 


\section{Diseño de interfaz Hinge con TOB y M1}

\subsection{Refinado de Mallado}

Explicar el objetivo del estudio de sensibilidad y el problema de los tornillos y la IF (REQUISITO DE UNION??)

\subsubsection{Malla 1}

Meter foto de la malla (Paint)
Explicar lo del rango de nodos del M1 ribs y la malla del punto de partida

\subsubsection{Malla 2}

Foto de la malla (Paint)
Cambios respecto a la malla 1

\subsubsection{Malla 3}

Repetir

\subsubsection{Malla 4}

Foto de la malla (Final)
Cambios y problema de las label -> Solución de cambiar label de geometria
Meter una labla con estructura de label??

\subsubsection{Resultados}

Meter tabla y gráficas del Delta T, para comparar mejor meter tabla con los puntos maximos.

\subsection{Efecto de valores hc en DeltaT I/F}

Explicar que se hace este otro analisis pero cambiando valor de hc, en la malla mas refinada y la original para ver el efecto combinado del cambio de mallado y hc simultaneamente.

\subsubsection{Malla 1-Gruesa}

Meter los 4 casos

\subsubsection{Malla 4-Fina}

Meter los 4 casos

\subsubsection{Resultados}

Comparar y explicar que el efecto de la malla es menor que el cambio de hc

\subsection{Conclusiones}

Resumen final de ambos analisis



\end{document}
